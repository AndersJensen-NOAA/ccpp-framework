\chapter{Introduction}\label{chap_introduction}
\setlength{\parskip}{12pt}

The Common Community Physics Package (CCPP) is designed to facilitate the implementation of physics innovations in state-of-the-art atmospheric models, the use of various models to develop physics, and the acceleration of transition of physics innovations to operational NOAA models. The CCPP consists of two separate software packages, the pool of CCPP-compliant physics schemes (\execout{ccpp-physics}) and the framework (driver) that connects the physics schemes with a host model (\execout{ccpp-framework}).

The connection between the host model and the physics schemes through the CCPP framework is realized with caps on both sides as illustrated in Fig.~\ref{fig_ccpp_design_with_ccpp_prebuild} in Chapter~\ref{chap_hostmodel}. While the caps to the individual physics schemes are auto-generated, the cap that connects the framework (Physics Driver) to the host model must be created manually. The CCPP framework generates a large fraction of code that can be included in the host model cap to facilitate this process. For more information about the CCPP design and implementation, see the CCPP Design Overview at {\url{https://dtcenter.org/gmtb/users/ccpp/docs/}}.

This document serves two purposes, namely to describe the technical work of writing a CCPP-compliant physics scheme and adding it to the pool of CCPP physics schemes (Chapter~\ref{chap_schemes}), and to explain in detail the process of connecting an atmospheric model (host model) with the CCPP (Chapter~\ref{chap_hostmodel}). For further information and an example for integrating CCPP with a host model, the reader is referred to the GMTB Single Column Model (SCM) User and Technical Guide v2.1 available at {\url{https://dtcenter.org/gmtb/users/ccpp/docs}}.

At the time of writing, the CCPP is supported for use with the GMTB Single Column Model (SCM). Support for use of CCPP with the experimental version of NCEP's Global Forecast System (GFS) that employs the Finite-Volume Cubed-Sphere dynamical core (FV3GFS) is available as an internal release for the developers. A public release of FV3GFS with CCPP is planned for early 2019.

The GMTB welcomes contributions to CCPP, whether those are bug fixes, improvements to existing parameterizations, or new parameterizations. There are two aspects of adding innovations to the CCPP: technical and programmatic. This Developer's Guide explains how to make parameterizations technically compliant with the CCPP. Acceptance in the master branch of the CCPP repositories, and elevation of a parameterization to supported status, depends on a set of scientific and technical criteria that are under development as part of the incipient CCPP Governance. Contributions can be made in form of git pull requests to the development repositories. Before initiating a major development for the CCPP please contact GMTB at \url{gmtb-help@ucar.edu} to create an integration and transition plan. For further information, see the Developer's Corner for CCPP at \url{https://dtcenter.org/gmtb/users/ccpp/developers/index.php}. Note that while the pool of CCPP physics and the CCPP framework are managed by the Global Model Test Bed (GMTB) and governed jointly with partners (e.g., NCAR), the code governance for the host models lies with their respective organizations. Therefore, inclusion of CCPP within those models should be brought up to their governing bodies.