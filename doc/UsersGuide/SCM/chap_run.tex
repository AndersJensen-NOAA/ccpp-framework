\chapter{Running SCM}
\label{chapter: runningscm}
\setlength{\parskip}{12pt}
\label{section: runningscm}

\section{A sample test case}

The CCPP uses a run-time, dynamically loaded physics library to select physics suites at runtime.  To utilize a given library (of CCPP-complient!) physics parameterizations, first append the path to the physics suite libraries that you need
within the CCPP. For example, for the gmtb-gfsphysics suite, use:
\begin{itemize}
	\item for sh, bash

     \exec{export LD\_LIBRARY\_PATH=\$\{LD\_LIBRARY\_PATH\}:/path/to/build/directory/ccpp-physics}\\

	\item for csh

     if appending: \\
	\exec{setenv LD\_LIBRARY\_PATH \$\{LD\_LIBRARY\_PATH\}:/path/to/build/directory/ccpp-physics}

     if setting: \\
	\exec{setenv LD\_LIBRARY\_PATH path/to/build/directory/ccpp-physics}
\end{itemize}


\section{Run the SCM with the supplied case} 
The test case provided with the SCM is TWP-ICE, the Tropical Warm Pool-International Cloud Experiment. The SCM will go through the time steps, applying forcing and calling the physics defined in the suite definition file.
There is a single command line argument required, which is the name of the case configuration file.  For the provided case, that name is \exec{twpice}.

  \exec{./gmtb\_scm twpice}

A netcdf output file is generated in the location specified in the case
configuration file. Any standard NetCDF file viewing or analysis tools may be used to 
examine the output file (ncdump, ncview, NCL, etc).  For the twpice case, it is located in:

\exec{gmtb-scm/output\_twpice/output.nc}

Additional details may be found in the SCM technical documention, including setting physics parameters, examining the output files, etc.  

\url{https://dtcenter.org/gmtb/users/ccpp/index.php}


\section{Setting up the physics suite}
First, a physics suite is defined using an XML file located in
src/ccpp/examples. The XML file contains the suite name, the number of suite
"parts" (suite parts exist so that an atmosphere "cap" can execute code between
calls to the physics driver), the number of subcycles for each scheme (if
physics schemes require smaller time steps than the dynamics), and the scheme
names. Scheme names found in the suite XML files must correspond to schemes
located within the physics directory.

Using a suite in the SCM framework involves specifying its name in the case
configuration file to be used. For example, for the twpice case
(case\_config/twpice.nml), the variable 'physics suite' is set to the desired
suite name. NOTE: As mentioned in the 'Running' section above, since the schemes
 are in their own libraries, you must specify the path to the compiled scheme
 libraries that are being used in the suite by appending the LD\_LIBRARY\_PATH
 (or DYLD\_LIBRARY\_PATH for Mac OS). Without this step, the scheme libraries will
  not be found at runtime.


